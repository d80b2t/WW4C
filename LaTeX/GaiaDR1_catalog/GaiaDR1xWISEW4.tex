\documentclass[usenatbib]{mn2e}
\input{format}


\begin{document}


\title[WISE W4s not in Gaia]
{Combing the WISE W4 catalogue and the
  Gaia Data Release 1: Searching for Infrared bright, optically faint
  objects across the full sky}
\author[Ross \& Hambly]
{Nicholas P. Ross$^{1}$\thanks{email: npross@roe.ac.uk}, 
  Nigel C. Hambly$^{1}$\\
$^1$Institute for Astronomy, Blackford Hill, Edinburgh, EH9 3HJ, United Kingdom \\ 
 % $^1$Institute for Astronomy, SUPA\footnote{Scottish Universities Physics Alliance}, University of Edinburgh, Royal Observatory, Edinburgh, EH9 3HJ, United Kingdom\\
}


\maketitle
\begin{abstract}
 Lorem ipsum dolor sit amet, consectetur adipiscing elit. Aliquam
porta sodales est, vel cursus risus porta non. Vivamus vel pretium
velit. Sed fringilla suscipit felis, nec iaculis lacus convallis
ac. Fusce pellentesque condimentum dolor, quis vehicula tortor
hendrerit sed. Class aptent taciti sociosqu ad litora torquent per
conubia nostra, per inceptos himenaeos. Etiam interdum tristique diam
eu blandit. Donec in lacinia libero.
%%
Sed elit massa, eleifend non sodales a, commodo ut felis. Sed id
pretium felis. Vestibulum et turpis vitae quam aliquam convallis. Sed
id ligula eu nulla ultrices tempus. Phasellus mattis erat quis metus
dignissim malesuada. Nulla tincidunt quam volutpat nibh facilisis
euismod. Cras vel auctor neque. Nam quis diam risus.
\end{abstract}
\begin{keywords}
galaxies: clustering -- luminous red galaxies: general -- cosmology: 
observations -- large-scale structure of Universe.
\end{keywords}



\section*{Preamble}
All the code and plotting packages, and smaller catalogues are
available at \href{https://github.com/d80b2t/WW4C}{{\tt
https://github.com/d80b2t/WW4C}}.  The .tex and PDF file can be found
\href{https://github.com/d80b2t/WW4C/tree/master/LaTeX/GaiaDR1_catalog}{{\tt
here}}.

\noindent
The four `obvious' files that aren't on the GitHub are:: \\
GaiaDR1xWISEw4\_10as\_noDupes\_sorted.csv (1.1G); \\
WISE\_W4\_DecOrdered.dat (1.4GB);\\
WISE\_W4\_cat\_with\_GaiaNull.csv  (2.6GB);\\
WISE\_W4\_cat\_with\_Gaia.csv  (3.2GB).\\



\section{Introduction}
Extremely Red Quasars \citep[ERQs; ][]{Ross15, Hamann17} are a unique obscured
quasar population with extreme physical conditions related to powerful
outflows across the line-forming regions \citep{Zakamska16,
Alexandroff17}.  These sources are the signposts of the most extreme
form of quasar feedback at the peak epoch of galaxy formation, and may
represent an active ``blow-out'' phase of quasar evolution.
%%
The energetics of luminous AGN, i.e. quasars, is thought to be a major
ingredient and physical process in galaxy formation and evolution, 
vital for suppressing, or ``quenching'' star-formation and leading 
to the population or red galaxies seen in the local Universe. 
%%
As such, a great deal of effort has gone into identifying the sites of
the so-called ``AGN feedback'' , \citep[e.g.,][]{CanoDiaz12, Page12,
Maiolino12, Cicone14, Harrison14, ZakamskaGreene14, Harrison17} and
\citep{Fabian12, KormendyHo13, HeckmanBest14, Netzer15, KingPounds15,
NaabOstriker17} for relevant reviews.
%%
Motivated by the ERQs, we are interested in finding further infrared bright, 
optically faint, high-$z$ quasars. 

The WISE W4 23$\mu$m band is shallow. Therefore, anything that is
detected in W4, and is {\it not} a Milky Way star, or nearby
e.g. dusty spiral galaxy, is going to be intrinsically very
luminous. Furthermore, objects that are also detected in W4, but are
faint/non-detected in the optical, or in the shorter WISE W1/W2 bands
also have very interesting properties, \citep[e.g.,][]{Assef15,
Tsai15, Lonsdale15, Assef16, Diaz-Santos16, Ricci17, Wu17, Jones17,
Farrah17}.  The WISE W4 band is also crucial in the discovery and
characterization of the ``Extremely Red Quasar'' population, using a
colour selection of $r_{\rm AB} − W4_{\rm Vega} >14.0$ \citep{Ross15,
Zakamska16, Hamann17}.
%%
The WISSH Quasars project \citep{Bischetti17, Duras17, Martocchia17}  
is also focusing on WISE/SDSS selected ``hyper-luminous'' broad-line
quasars at $z\approx 1.5 - 5$. 

\citet{Singal16}     %% http://adsabs.harvard.edu/abs/2016ApJ...831...60S
\citet{LaMassa17} %% http://adsabs.harvard.edu/abs/2017ApJ...847..100L
\citet{Toba15}       %% http://adsabs.harvard.edu/abs/2015PASJ...67...86T
\citet{Toba16}       %% http://adsabs.harvard.edu/abs/2016ApJ...820...46T
\citet{Toba17}       %%  http://adsabs.harvard.edu/abs/2017ApJ...835...36T


Note, the W4 channel effective wavelength was recalibrated from the
original 22$\mu$m by \citet{Brown14b}.


\section{Matching the Gaia DR1 and WISE W4 catalogs}

    \subsection{WISE}
    The Wide-field Infrared Survey Explorer (WISE) mission description and
    initial on-orbit performance is described in \citet[][]{Wright10}.
    We use the \href{http://wise2.ipac.caltech.edu/docs/release/allwise/expsup/}
    {\tt AllWISE Data Release}. 

    There are 40,939,966 objects, with $>2\sigma$ detections at WISE
    W4 in the AllWISE Catalog. The W4 PSF is 12'', but the centroid
    positions should be good to $\lesssim2''$ (WISE HelpDesk,
    priv. comm.).  Figure~\ref{fig:fig1} shows the sky distribution of
    these 40.9M objects.  The WISE scanning pattern can clearly be 
    \href{http://wise2.ipac.caltech.edu/docs/release/allwise/expsup/sec4_2.html}{{\tt
      seen}}.

    \subsection{Gaia DR1} 
    We use the \href{https://www.cosmos.esa.int/web/gaia/dr1}{{\tt Gaia Data Release 1}}.
    \citep{Gaia16a, Gaia16b}\footnote{I don't know how to fix this citation.}
%    \citep{Prusti16, Brown16}.
    % Gaia Collaboration, Prusti, T., de Bruijne, J. H. J., et al. 2016, A & A, 595, A1
    % Gaia Collaboration, Brown, A. G. A., Vallenari, A., et al. 2016, A & A, 595, A2
    % The Gaia mission Gaia Collaboration, Prusti, T., de Bruijne, J.H.J., et al., 2016a (arXiv)
    % Gaia Data Release 1: Summary of the astrometric, photometric, and survey properties Gaia Collaboration, Brown, A.G.A., Vallenari, A., et al., 2016b (arXiv)
    %% (What about the ten or so other Gaia Data Release 1 papers??!!) 
    %% Gaia Data Release 1: On-orbit performance of the Gaia CCDs at L2 Crowley et al. (arXiv) 
    %% Gaia Data Release 1: Pre-processing and source list creation Fabricius et al. (arXiv)
    %% Gaia Data Release 1: Astrometry: one billion positions, two million proper motions and parallaxes Lindegren et al. (arXiv)
    %% Gaia Data Release 1: The photometric data van Leeuwen et al. (arXiv)
    %% Gaia Data Release 1: Principles of the Photometric Calibration of the G band Carrasco et al. (arXiv)
    %% Gaia Data Release 1: Validation of the Photometry Evans et al. (arXiv)
    %% Gaia Data Release 1: Catalogue validation Arenou et al. (arXiv)
    %% Gaia Data Release 1: The reference frame and the optical properties of ICRF sources Mignard et al. (arXiv)
    %% Gaia Data Release 1: The variability processing & analysis and its application to the south ecliptic pole region Eyer et al. (arXiv)
    %% We do not use Marrese et al.  Gaia Data Release 1: Cross-match with external catalogues - algorithm and statistics

    The Gaia DR1
    \href{https://www.cosmos.esa.int/documents/29201/1125416/magnitude+histogram+placeholder.png/}
    {magnitude histogram} peaks around 20th magnitude in G-band. The
    G-band is a very wide filter, that covers the wavelength range frthat
    covers the wavelength range from about 350 to 1000nm, with the maximum
    energy transmission at $\sim$715 nm and the full width at half maximum
    of 408 nm.  (Jordi \&. Carrasco, 2007, ASPC, 364, 215).
    
    There are 1,142,679,769 sources in total in the Gaia DR1. 



    \subsection{Matching}
    We match the two catalogues with a matching radius of 2'' and 10''. 
    This is done using some v. nice and quick code from R. Collins [More details required here]. 
    The results are given in Figure~\ref{fig:fig2} and Figure~\ref{fig:fig4}. 
    Figure~\ref{fig:fig3} shows the WISE W4 Unique ID (UID; which isa a proxy for object declination) 
    versus the Gaia DR1 Source ID. 


\section{Results of matching}
The matched 10'' GaiaDR1xWISEW4 catalogue returns 71,593,922 objects
since one-to-many (WISE-to-Gaia) matches are allowed. The Python
\href{https://docs.scipy.org/doc/numpy/reference/generated/numpy.unique.html}{{\tt
numpy unique}} command is used to return the sorted unique elements of
this catalogue. 24,671,865 WISE W4 objects have a unique match with a
Gaia source within 10''. 

Thus, 16,268,101 objects in the WISE W4 catalogue do not have a

Figure~\ref{fig:fig4} shows the matching radius separation histograms for objects, 
with and without duplicates. 

\begin{table}
\begin{centering}
\begin{tabular}{l r r l l }
\hline
\hline
%  \multicolumn{1}{|c|}{Name} &   \multicolumn{1}{c|}{Mean} &   \multicolumn{1}{c|}{SD} &   \multicolumn{1}{c|}{Minimum} &   \multicolumn{1}{c|}{Maximum} &   \multicolumn{1}{c|}{nGood} \\
  Name &  Mean & $\sigma$ & Min. & Max. \\
\hline
  W4\_no   & 2.0469982E7 & 1.181835E7 & 0 & 40939965 \\
  RA           & 198.89731 & 98.82655 & 1.9600E-5 & 359.9998663 \\
  Decl         & -4.5770774 & 41.878914 & -89.9769927 & 89.9673322 \\ 
  uid           & 2.4711034E7 & 2.592262E7 & -1.0 & 7.1593991E7 \\
  Gaia\_ID  & 2.42485497E18 & 2.43082065E18 & -1.0 & 6.9175287184039117E18 \\
  radius      & -0.35169324 & 0.52868944 & -1.0 & 0.16666666 \\
\hline
\hline
\end{tabular}
      \label{tab:stats}
\caption{WISE\_W4\_cat\_with\_Gaia. N$_{\rm Good}=40939966$.}
\end{centering}
\end{table}








\begin{figure*}
%    \centering
    \includegraphics[height=10.0cm,width=16.0cm]
    {../../Gaia/plots/WISE_W4_DetBitge8_Aittoff_Galactic.png}
    \caption[The all-sky distribution of the 40.9M objects in the AllWISE W4 catalog.]
    {The all-sky distribution of the 40.9M objects in the AllWISE W4 catalog.
    The WISE scanning pattern can clearly be seen.}
    \label{fig:fig1}
\end{figure*}


\begin{figure}
   \centering
  \includegraphics[height=8.0cm,width=8.0cm]
 {../../Gaia/plots/GaiaDR1xWISEW4_2as_histo.png}
    \caption[]
    {The matching radius separation histograms for objects, when a 2'' matching radius is applied.}
    \label{fig:fig2}
\end{figure}

\begin{figure}
    \centering
    \includegraphics[height=8.0cm,width=8.0cm]
    {../../Gaia/plots/GaiaDR1xWISEW4_10asDupes_histo.png}
    \caption[The matching radius separation histograms for objects, with and without duplicates.]
    {The matching radius separation histograms for objects, when a 10'' matching radius is applied, with and without duplicates.}
    \label{fig:fig3}
\end{figure}

\begin{figure}
    \centering
    \includegraphics[height=8.0cm,width=8.0cm]
    {../../Gaia/plots/Gaia_sourceID_vs_WW4C_uid.png}
    \caption[The WISE W4 Unique ID (UID; which isa a proxy for object declination) versus the Gaia DR1 Source ID.]
    {The WISE W4 Unique ID (UID; which isa a proxy for object declination) versus the Gaia DR1 Source ID.}                    
    \label{fig:fig4}
\end{figure}



Figure~\ref{fig:fig5} (top) shows the all-sky distribution for the
full 40.9M objects that are detected in WISE W4 (same as
Figure~\ref{fig:fig1} but with a different
colour-scale). Figure~\ref{fig:fig5} (middle) shows the all-sky distrubtion of 
objects that were matched to a Gaia DR1 source. The overdensity of the Milky 
Way is clearly seen. Figure~\ref{fig:fig5} (bottom) shows the 16.3M objects 
that do {\it not} have a match in the Gaia DR1. 


\newpage
\centering
\begin{figure*}
\begin{center}
    \includegraphics[height=7.5cm,width=14.0cm]{../../Gaia/plots/all_rainbow3.png}
    \includegraphics[height=7.5cm,width=14.0cm]{../../Gaia/plots/matches_rainbow3.png}
    \includegraphics[height=7.5cm,width=14.0cm]{../../Gaia/plots/nonmatches_rainbow3.png}
    \caption[Lorem ipsum dolor sit amet, consectetur adipiscing elit. Aliquam porta sodales est, vel cursus risus porta non.]
    {Three aitoff projects for the Gaia DR1$\times$WISE W4 matched catalog.
    {\it Top:} The full 40.9M objects that are detected in WISE W4. 
    {\it Middle:}  The 24.7M WISE W4 objects that are matched to a Gaia DR1 source. 
    {\it Bottom:}  The 16.3M WISE W4 objects that are not matched to a Gaia DR1 source. 
  }
    \label{fig:fig5}
\end{center}
\end{figure*}






\bibliographystyle{mn2e}
\bibliography{/cos_pc19a_npr/LaTeX/tester_mnras}


\end{document}